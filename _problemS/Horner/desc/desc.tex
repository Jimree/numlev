\documentclass[12pt]{extarticle}
\usepackage[utf8]{inputenc}
\usepackage{amsmath}
\usepackage{verbatim}
\usepackage{setspace}
\onehalfspace




\begin{document}

\centerline{\bf Horner módszer }
\noindent
$p(x)=p_{n}x^{n}+\ldots +p_{1}x+p_{0}$ kiszámolása különböző $x$-ekre.


\noindent
{\bf Input}
\begin{flalign*}
& n&\\
& p_{n}\: p_{n-1}\ldots p_{1}\: p_{0}\\
& m\\
& x_{1}\: x_{2} \ldots x_{m}
\end{flalign*}


\noindent
{\bf Output}
\begin{flalign*}
& p(x_{1})\: p(x_{2})\ldots p(x_{m})&
\end{flalign*}


\noindent
{\bf Korlátok}\newline
$0<n<100$, $m<100.$  A kiírt számok {\bf 12} értékes jegyet tartalmazzanak!



\noindent
{\bf PéldaInput}
\verbatiminput{../io/in1}

\noindent
{\bf PéldaOutput}
\verbatiminput{../io/out1}


\end{document}
