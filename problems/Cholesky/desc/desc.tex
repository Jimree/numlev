\documentclass[12pt]{extarticle}
\usepackage[utf8]{inputenc}
\usepackage{amsmath}
\usepackage{verbatim}
\usepackage{setspace}
\onehalfspace


\begin{document}

\centerline{\bf Cholesky felbontás }
\noindent Ha a szimmetrikus $A$-nak létezik a Cholesky-felbontása 
akkor írjuk ki az alsó háromszög alakú faktorát, egyébként a 
\texttt{fail} sztringet. A mátrix alsó háromszögrésze az input.


\noindent
{\bf Input}
\begin{flalign*}
& n &\\
& a_{11}\\
& a_{21}\: a_{22}\\
& \ldots \\
& a_{n 1}\ldots a_{n n}\\
\end{flalign*}


\noindent
{\bf Output}
\begin{flalign*}
& q_{11}&\\
& q_{21}\: q_{22}&\\
& \ldots& \\
& q_{n 1}\ldots q_{n n}&\\
\end{flalign*}


\noindent
{\bf Korlátok}\newline
$1<n<100$. 
A kiírt számok {\bf 12} értékes jegyet tartalmazzanak!



\noindent
{\bf PéldaInput}
\verbatiminput{../io/in29}

\noindent
{\bf PéldaOutput}
\verbatiminput{../io/out29}


\end{document}
