\documentclass[12pt]{extarticle}
\usepackage[utf8]{inputenc}
\usepackage{amsmath}
\usepackage{verbatim}
\usepackage{setspace}
\onehalfspace



\begin{document}

\centerline{\bf Fixpont-iteráció }
\noindent Az $F$ függvény fixpontját közelítjük $x_0$ pontból indulva,
 legfeljebb $s$ lépést végezve. Az iterációt hagyjuk abba a $k$. lépésnél 
ha azt tapasztaljuk hogy 
\begin{equation}
\label{Mctr}
\tag{Mctr}
|x_{k}-x_{k-1}|<|x_{k-1}-x_{k-2}|<\ldots <|x_{k-M+1}-x_{k-M}| 
\end{equation}
\noindent azaz, folyamatos $M$ hosszú csökkenés látható a szomszédos elemek távolságában.
Ha $x_k=\pm\infty,\texttt{Nan}$ vagy \textit{\eqref{Mctr}} nem teljesült, 
akkor a \texttt{fail} sztringet, egyébként $x_k$-t írjuk ki.
\[ 
F(x)=C_0*e^{C_1 x}+C_2 \sin(C_3 x)+C_4 \cos(C_5 x)+C_6 \sin(e^{C_7 x})
\]

\noindent
{\bf Input}
\begin{flalign*}
& C_0\ldots C_7\: &\\
& x_{0}\: s\: M&\\
\end{flalign*}


\noindent
{\bf Output}
\begin{flalign*}
& \text{lásd a leírást} &
\end{flalign*}


\noindent
{\bf Korlátok}\newline
$0<2*M<S<100.$
A kiírt számok {\bf 12} értékes jegyet tartalmazzanak!



\noindent
{\bf PéldaInput}
\verbatiminput{../io/in1}

\noindent
{\bf PéldaOutput}
\verbatiminput{../io/out1}


\end{document}
