\documentclass[12pt]{extarticle}
\usepackage[utf8]{inputenc}
\usepackage{amsmath}
\usepackage{verbatim}
\usepackage{setspace}
\onehalfspace


\begin{document}

\centerline{\bf Felező-módszer }
\noindent Az $F$ függvény gyökét közelítjük az $[a,b]$ intervallumon, amíg az intervallum hosszabb mint $\varepsilon$. 
Ha $F(a)*F(b)>0$ akkor a \texttt{fail} sztringet egyébként a végső intervallum végpontjait írjuk ki.
\[ 
F(x)=C_0*e^{C_1 x}+C_2 \sin(C_3 x)+C_4 \cos(C_5 x)+C_6 \sin(e^{C_7 x})
\]

\noindent
{\bf Input}
\begin{flalign*}
& C_0\ldots C_7\: &\\
& a\: b\: \varepsilon&\\
\end{flalign*}


\noindent
{\bf Output}
\begin{flalign*}
& \text{lásd a leírást} &
\end{flalign*}


\noindent
{\bf Korlátok}\newline
$0<\varepsilon.$
A kiírt számok {\bf 12} értékes jegyet tartalmazzanak!



\noindent
{\bf PéldaInput}
\verbatiminput{../io/in1}

\noindent
{\bf PéldaOutput}
\verbatiminput{../io/out1}


\end{document}
