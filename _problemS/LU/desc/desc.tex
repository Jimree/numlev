\documentclass[12pt]{extarticle}
\usepackage[utf8]{inputenc}
\usepackage{amsmath}
\usepackage{verbatim}
\usepackage{setspace}
\onehalfspace



\begin{document}

\centerline{\bf Mátrix LU felbontása }
\noindent Egy $n\times n$-es $A=(a_{ij})$ mátrix $LU$ felbontását kell kiszámolni. 
A sorcsere {\it nem} megengedett! Ennél a feladatnál nem kell abbahagyni a felbontás 
létrehozását ha $a_{ii}\approx 0$, csak akkor ha alatta van nemnulla elem! Az $LU$ felbontást egy mátrixban 
számoljuk és írjuk ki. Ha nincs felbontás, a \texttt{fail} sztring kerül az outputba.


\noindent
{\bf Input}
\begin{flalign*}
& n &\\
& a_{11}\ldots a_{1 n}&\\
& \ldots & \ldots\\
& a_{n 1}\ldots a_{n n}&\\
\end{flalign*}


\noindent
{\bf Output}
\begin{flalign*}
& \text{lásd a leírást} &
\end{flalign*}


\noindent
{\bf Korlátok}\newline
$0<n<100.$ 
A kiírt számok {\bf 12} értékes jegyet tartalmazzanak!



\noindent
{\bf PéldaInput}
\verbatiminput{../io/in14}

\noindent
{\bf PéldaOutput}
\verbatiminput{../io/out14}


\end{document}
